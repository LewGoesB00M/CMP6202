% -------------------------------------------------------------------------------
% Establish page structure & font.
\documentclass[12pt]{report}

\usepackage[total={6.5in, 9in},
	left=1in,
	right=1in,
	top=1in,
	bottom=1in,]{geometry} % Page structure

\usepackage{graphicx} % Required for inserting images
\graphicspath{{images/}} % Any additional images I use (BCU logo, etc) are from here.

\usepackage[utf8]{inputenc} % UTF-8 encoding
\usepackage[T1]{fontenc} % T1 font
\usepackage{float}  % Allows for floats to be positioned using [H], which correctly
                    % positions them relative to their location within my LaTeX code.
\usepackage{subcaption}

% -------------------------------------------------------------------------------
% Declare biblatex with custom Harvard BCU styling for referencing.
\usepackage[
    useprefix=true,
    maxcitenames=3,
    maxbibnames=99,
    style=authoryear,
    dashed=false, 
    natbib=true,
    url=false,
    backend=biber
]{biblatex}

% Additional styling options to ensure Harvard referencing format.
\renewbibmacro*{volume+number+eid}{
    \printfield{volume}
    \setunit*{\addnbspace}
    \printfield{number}
    \setunit{\addcomma\space}
    \printfield{eid}}
\DeclareFieldFormat[article]{number}{\mkbibparens{#1}}

% Declare it as the bibliography source, to be called later via \printbibliography
\addbibresource{report.bib}

% -------------------------------------------------------------------------------
% To prevent "Chapter N" display for each chapter
\usepackage[compact]{titlesec}
\usepackage{wasysym}
\usepackage{import}

\titlespacing*{\chapter}{0pt}{-2cm}{0.5cm}
\titleformat{\chapter}[display]
{\normalfont\bfseries}{}{0pt}{\Huge}

% -------------------------------------------------------------------------------
% Custom macro to make an un-numbered footnote.

\newcommand\blfootnote[1]{
    \begingroup
    \renewcommand\thefootnote{}\footnote{#1}
    \addtocounter{footnote}{-1}
    \endgroup
}

% -------------------------------------------------------------------------------
% Fancy headers; used to show my name, BCU logo and current chapter for the page.
\usepackage{fancyhdr}
\usepackage{calc}
\pagestyle{fancy}

\setlength\headheight{37pt} % Set custom header height to fit the image.

\renewcommand{\chaptermark}[1]{%
    \markboth{#1}{}} % Include chapter name.


% Lewis Higgins - ID 22133848           [BCU LOGO]                [CHAPTER NAME]
\lhead{Lewis Higgins - ID 22133848~~~~~~~~~~~~~~~\includegraphics[width=1.75cm]{BCU}}
\fancyhead[R]{\leftmark}

% -------------------------------------------------------------------------------

\title{CMP6202 Report}
\author{Lewis Higgins - Student ID 22133848}
\date{December 2024}

% -------------------------------------------------------------------------------

\begin{document}


\makeatletter
\begin{titlepage}
    \begin{center}
        \includegraphics[width=0.7\linewidth]{BCU}\\[4ex]
        {\large \bfseries  \@title }\\[2ex]
        {\large \bfseries  DRAFT VERSION }\\[2ex]
        {\@author}\\[30ex]
        {Word count: XXXX}\\[20ex]
    \end{center}
\end{titlepage}
\makeatother
\thispagestyle{empty}
\newpage


% Page counter trick so that the contents page doesn't increment it.
\setcounter{page}{0}

\tableofcontents
\thispagestyle{empty}

% Declaring un-numbered chapter because I prefer how it looks.
\chapter*{Introduction}
% Add it to the contents, because un-numbered chapters aren't by default.
\addcontentsline{toc}{chapter}{Introduction}
% Put the chapter name in the header.
\markboth{Introduction}{}

Your EDA can be very extensive, and you could potentially have pages and pages and pages of it; this isn't a bad thing.
The vast majority of any ML-related work is EDA because it gives you the background information on the dataset to then apply
when training the model, such as the identification of non-numerical columns and encoding them into numerical equivalents where
possible so that they become useful training data for the model, as ML models cannot interpret strings.
\\
\noindent
You should work with at least two ML algorithms. Compare them, pros/cons, why you settled on the one you did.
F1-Score if you do a classification problem, otherwise $r^2$. A template exists for both the presentation and report.
\noindent \textbf{You need to pay close attention to section 6.2, as he says it'll be at least 10\%}.
He mentions things like certificates, a linkedin profile and more can be relevant to this section.
Sounds less like "individual learning reflection" and more "individual reflection".
\\
\noindent 
You might be missing some \LaTeX packages, consult your lit review for the ones you might need.
This also applies to the CMP6230 report.
\\
\noindent
Kaggle can give you certificates, as can GitHub. With Github, it could be 
advisable to make this repo public, or perhaps a sanitised duplicate. 

\chapter{Notes, remove before final}
\section{Week 7}
\subsection{Naive Bayes}
This symbol | means "given that".
Sentiment analysis, this could be linked to your dissertation too.
Used in classification. After the formula, the larger number is chosen as the class (i.e. if spam scored 0.05 but good showed 0.12, good wins).
Gaussian NB is best used for continuous data and text data classification.
The prior probability is calculated by dividing the number of positive instances by the total number of instances.
When classifying spam emails, the probability of the email being spam is the prior probability.

\subsubsection{Cross Validation}
Train_test_split has some issues, due to its random selection of data to put in each split, which are addressed by cross validation.
Cross validation helps to reduce bias in an ML model.
\subsubsection{K-Fold}
Divides the dataset into "K" subsets/folds, then trains and evaluates the model "K" times, using a different fold as the test set while the others 
act as the test data.
\subsubsection{Stratified K-Fold}
A variation of K-Fold that ensures class balance, helping where the dataset is imbalanced. \textbf{Could be very relevant.}
Week 7 PPT slides 17 through 20 help with this and regular K-Fold.
\subsubsection{Accuracy score}
Your accuracy score will differ from the cross-validation score because it doesn't account for the averaged and varied subsets
used in cross-validation.

\subsection{Decision tree}
Another classification method, currently don't know much about it.
Initialised in code via DecisionTreeClassifier().


\end{document}